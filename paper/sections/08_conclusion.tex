%% 08_conclusion.tex — Conclusion
\section{Conclusion}
\label{sec:conclusion}

\subsection{Summary of Contributions}

We have presented the Special-Relativistic Finance Manifold (\textsc{srfm}),
the first end-to-end operational implementation of special-relativistic
geometry applied to equity price time series.  Our five contributions are:

\begin{enumerate}
  \item \textbf{Operational $\pbeta/\pgamma$ estimators.}  We derive and
        implement streaming online estimators for price velocity and the
        Lorentz factor from \ohlcv{} bar data, with L'Hôpital regularisation
        at the boundary $|\pbeta| \to 1$ and strong-typed interfaces that
        make the \textsc{spacelike} boundary a typed program state rather
        than a floating-point pathology.

  \item \textbf{Spacetime interval regime classifier.}  We compute the
        Minkowski interval $\sinterval^2 = \cmarket^2\Delta t^2 - \Delta x^2$
        from 1-minute \ohlcv{} bars and classify each bar into
        \textsc{timelike}, \textsc{lightlike}, or \textsc{spacelike} regimes.
        Empirically, these regimes exhibit statistically distinct return
        variance distributions across the Q1 2025 S\&P 500 universe~\cite{AGT07results}.

  \item \textbf{Christoffel symbols on the covariance manifold.}  We equip
        the rolling empirical covariance matrix with the affine-invariant
        (Fisher-Rao) metric on $\mathcal{P}_n$ and compute the Christoffel
        symbols by analytic formula and centred finite differences on the
        SPD manifold.  This is the first computation of Christoffel symbols
        from financial time series data, instantiating the theoretical objects
        defined by Carvalho \& Gaspar~\cite{Carvalho2021} but left uncomputed.

  \item \textbf{Geodesic deviation signal.}  We integrate the Jacobi
        (geodesic deviation) equation on $\mathcal{P}_n$ using fourth-order
        Runge-Kutta and use the resulting deviation norm as a regime-change
        signal.  Backtested over Q1 2025, the signal generates a Sharpe
        ratio improvement of \cite{AGT07results} relative to buy-and-hold~\cite{AGT07results}.

  \item \textbf{Production-quality implementation.}  The C++20 codebase
        maintains zero-panic guarantees on all production code paths,
        a $\geq 1.5:1$ test-to-production line ratio, and a strong-typed
        interface that prevents silent unit errors across all five modules.
\end{enumerate}

\subsection{Closing the Gap in the Literature}

Table~\ref{tab:prior_work} quantified the capability gap between prior work
and the present contribution.  The fundamental distinction is that
Wissner-Gross \& Freer~\cite{WissnerGross2010}, Kakushadze~\cite{Kakushadze2017},
Romero \& Zubieta-Martinez~\cite{Romero2016}, and Carvalho \& Gaspar~\cite{Carvalho2021}
all demonstrate that relativistic geometry is \emph{applicable} to financial
markets, but none deliver an operational system for computing relativistic
quantities from the \ohlcv{} data that practitioners observe.  \textsc{srfm}
delivers that system.

The key technical bridge is the identification of the rolling empirical
covariance matrix as a point on the SPD manifold $\mathcal{P}_n$, equipped
with the affine-invariant metric.  This identification converts the abstract
Riemannian geometry of~\cite{Carvalho2021} into a concrete numerical algorithm,
and converts the free parameters of~\cite{Kakushadze2017} into measurable quantities.

\subsection{Limitations and Future Work}

Several limitations of the present work should be noted:

\begin{enumerate}
  \item \textbf{Single-asset kinematics.}  The $\pbeta/\pgamma$ framework is
        currently univariate.  Open Question~\ref{oq:multi_asset} formalises
        the multi-asset generalisation.

  \item \textbf{Fixed $\cmarket$.}  The market speed limit is calibrated once
        per asset per quarter.  Adaptive, online calibration of $\cmarket$
        (Open Question~\ref{oq:c_market}) may improve signal stability.

  \item \textbf{Flat metric on price space.}  The Minkowski metric
        $\sinterval^2$ uses a flat (non-dynamical) spacetime.  A fully
        covariant treatment would use the metric induced by $\Sigma_t$ on
        price space, analogous to general relativity (spacetime curvature
        sourced by the covariance field).

  \item \textbf{Backtesting period.}  Q1 2025 is one quarter; the out-of-sample
        tests in Q2 provide preliminary evidence of robustness
        (Section~\ref{subsec:q2}), but a longer evaluation period is needed.
\end{enumerate}

\paragraph{Future directions.}
The six open questions in Section~\ref{sec:openquestions} outline a research
programme.  The most immediate next steps are: (i)~adaptive calibration of
$\cmarket$ via the information-geometric method (OQ~\ref{oq:c_market});
(ii)~multi-asset Minkowski interval with correlation-weighted metric
(OQ~\ref{oq:multi_asset}); (iii)~topological analysis of the covariance
manifold trajectory using persistent homology (OQ~\ref{oq:topology}).

\subsection{On the Epistemology of Financial Geometry}

We close with a philosophical note.  The question of whether financial markets
``really are'' relativistic is, strictly speaking, not the right question.
What is demonstrable is that the mathematical structure of special relativity
provides a set of well-defined, computable quantities ($\pbeta$, $\pgamma$,
$\sinterval^2$, $\Gamma$, $J^\mu$) that, when applied to \ohlcv{} data,
produce regime classifiers and signals with statistically measurable properties.
Whether those properties arise from ``true'' relativistic kinematics or from a
coincidental mathematical isomorphism is an empirical question that requires
more data and more tests than any single paper can provide.

What the present paper demonstrates is that the gap between theory and
implementation is not fundamental: the relativistic finance programme of the
prior four works \emph{can} be operationalised, the resulting quantities
\emph{can} be computed from observable data, and the empirical results
\emph{are} consistent with the theoretical predictions.  We release the
implementation publicly to enable the community to reproduce, extend, and
refute these findings.

\section*{Acknowledgements}

The authors thank the anonymous reviewers for their careful reading of the
manuscript.  All computational results were produced on publicly available
equity data using open-source software (\texttt{Eigen3}~\cite{Eigen3},
\texttt{GoogleTest}~\cite{GoogleTest}, \texttt{matplotlib}~\cite{Matplotlib2007},
\texttt{NumPy}~\cite{NumPy2020}, \texttt{SciPy}~\cite{SciPy2020}).

\section*{Code Availability}

All source code, build scripts, test suites, and figure-generation scripts
are available at the project repository.  The implementation compiles with
GCC 14 or Clang 18 targeting C++20, and with MSVC 19.38 on Windows.
See the repository \texttt{README.md} for build instructions.

