%% 07_open_questions.tex — Open Questions
\section{Open Questions}
\label{sec:openquestions}

The \textsc{srfm} framework raises several foundational questions that we leave
open.  We formalise each as a precise mathematical statement.

\begin{openquestion}[Uniqueness of the Lorentz Group Action]
\label{oq:lorentz_unique}
\emph{Is the Poincaré group the unique symmetry group consistent with the
efficient market hypothesis (EMH)?}

More precisely, let $\mathcal{F}$ be the space of all causal price processes
(adapted to the filtration generated by public information) and let
$G$ be a Lie group acting on $\mathcal{F}$ that preserves the no-arbitrage
condition.  The EMH in its weak form states that the price process is a
martingale under the risk-neutral measure $\mathbb{Q}$.

\textbf{Question.} Is $G \cong \mathrm{ISO}(1,1)$ (the Poincaré group in
1+1 dimensions, i.e., translations and Lorentz boosts) the minimal group
satisfying both (i) the martingale condition under $\mathbb{Q}$ and (ii)
the bounded-propagation constraint $|\pbeta| \leq 1$?

This would provide a derivation of the relativistic framework from
first principles rather than an analogy, and would establish whether
other symmetry groups (e.g., conformal, de Sitter) are admissible.
\end{openquestion}

\begin{openquestion}[Generalisation to Many Assets]
\label{oq:multi_asset}
\emph{What is the correct multi-asset generalisation of the spacetime interval?}

For a single asset with price velocity $\pbeta$, the interval is
$\sinterval^2 = \cmarket^2\Delta t^2 - \Delta x^2$.  For a universe of $N$
assets with price velocities $\pbeta^{(i)}$, $i = 1, \ldots, N$, two
natural generalisations exist:

\begin{enumerate}
  \item \textbf{Product metric:} $\sinterval^2 = \cmarket^2\Delta t^2 -
        \sum_i (\Delta x^{(i)})^2$, treating the price vector as a
        multi-dimensional displacement.

  \item \textbf{Correlation-weighted metric:} $\sinterval^2 = \cmarket^2\Delta t^2 -
        \boldsymbol{\Delta x}^\top \Sigma^{-1} \boldsymbol{\Delta x}$,
        where $\Sigma$ is the correlation matrix and the interval is the
        Mahalanobis distance from the origin.
\end{enumerate}

\textbf{Question.} Which generalisation preserves the causal structure of the
market (i.e., which regime classification best predicts return variance
conditioning properties analogous to those in Section~\ref{subsec:q1}) for
$N > 1$?  Is there a third generalisation with better empirical properties?
\end{openquestion}

\begin{openquestion}[The Calibration of $\cmarket$]
\label{oq:c_market}
\emph{Is there a first-principles determination of $\cmarket$, or is it
necessarily an empirical parameter?}

This is arguably the most important open question in the framework.  The
market speed limit $\cmarket$ determines which bars are \textsc{timelike}
and which are \textsc{spacelike}, and thus controls the regime classifier
(Section~\ref{subsec:interval}).  Three candidate calibration methods are:

\begin{enumerate}
  \item \textbf{Percentile heuristic:} $\cmarket = Q_{99.5}(|\overline{r}_t|)$
        over a trailing window.  Simple and adaptive; but the percentile
        threshold is itself a free parameter.

  \item \textbf{Maximum likelihood:} Fit a model
        $r_t | \text{regime} \sim p_{\textsc{tl}}$ or $p_{\textsc{sl}}$
        and maximise the joint likelihood over $\cmarket$ and the mixture
        weights.  Principled but computationally expensive.

  \item \textbf{Information-geometric:} Choose $\cmarket$ to maximise the
        KL divergence $D_{\mathrm{KL}}(p_{\textsc{tl}} \| p_{\textsc{sl}})$,
        i.e., the calibration that most separates the two regime distributions.
        This is equivalent to maximising the Fisher information about regime
        membership.
\end{enumerate}

\textbf{Formal question.}  Does there exist a $\cmarket^* = \cmarket^*(\mathcal{P})$
that is a functional of the price process $\mathcal{P}$ alone (not depending
on arbitrary threshold choices), such that the resulting regime classification
is invariant under diffeomorphisms of the price axis?  If so, what is its form?
\end{openquestion}

\begin{openquestion}[Quantum Corrections to Geodesic Deviation]
\label{oq:quantum_corrections}
\emph{Does the geodesic deviation signal admit a path-integral formulation?}

The Jacobi equation~\eqref{eq:jacobi_eq} is the classical (deterministic)
equation of geodesic deviation on $\mathcal{P}_n$.  In the Romero-Zubieta
quantum finance framework~\cite{Romero2016}, the price spinor $\Psi$ evolves
via the Dirac equation, and classical trajectories are replaced by path integrals.

\textbf{Question.}  Is there a path-integral formulation
\begin{equation}
  \langle J^\mu(\tproper) \rangle
  = \int \mathcal{D}[\gamma] \, J^\mu[\gamma] \, e^{iS[\gamma]/\hbar_{\mathrm{mkt}}}
\end{equation}
over paths $\gamma$ on $\mathcal{P}_n$, where $\hbar_{\mathrm{mkt}}$ is a
``market Planck constant'' encoding the minimum uncertainty in price-time
resolution (e.g., the tick size times the minimum bar duration)?  If so, what
is the quantum correction to the classical Jacobi field, and is it measurable
at current market microstructure resolution?
\end{openquestion}

\begin{openquestion}[Validity of the Information Ratio under $\pgamma$ Scaling]
\label{oq:ir_gamma}
\emph{Does the Sharpe ratio transform covariantly under Lorentz boosts?}

Define the information ratio in frame $\mathcal{O}$ as
$\mathrm{IR} = \mu / \sigma$, where $\mu$ and $\sigma$ are the mean and
standard deviation of returns per wall-clock bar.  Under a boost to frame
$\mathcal{O}'$ with velocity $\pbeta$:
\begin{itemize}
  \item Returns transform as $r' = r / \pgamma$ (time dilation; fewer returns
        per proper-time bar);
  \item Volatility transforms as $\sigma' = \sigma / \pgamma$
        (Eq.~\ref{eq:kakushadze_vol}, following Kakushadze~\cite{Kakushadze2017}).
\end{itemize}
Naively, $\mathrm{IR}' = (\mu/\pgamma) / (\sigma/\pgamma) = \mu/\sigma = \mathrm{IR}$.

\textbf{Question.}  Is the information ratio Lorentz-invariant?  If so,
this provides a new no-arbitrage argument: if a trading strategy has
$\mathrm{IR} > 0$ in one frame, it has $\mathrm{IR} > 0$ in all frames,
consistent with the absence of risk-free arbitrage.  However, if the
mean $\mu$ and $\sigma$ do not transform identically (e.g., if $\mu$
is affected by the risk-free rate and $\sigma$ is not), the invariance
breaks down.  Characterise the conditions under which $\mathrm{IR}$ is
frame-invariant.
\end{openquestion}

\begin{openquestion}[Topology of the Market Manifold]
\label{oq:topology}
\emph{What is the global topology of the market covariance manifold, and
does it change at phase transitions?}

Locally, the covariance manifold $\mathcal{P}_n$ is topologically trivial
(diffeomorphic to $\mathbb{R}^{n(n+1)/2}$).  However, the \emph{effective}
manifold traced by rolling covariance matrices $\{\Sigma_t\}_{t=1}^{T}$
may have non-trivial topology: it may cluster into multiple connected
components corresponding to distinct market regimes, and transitions between
components may correspond to topological events (``phase transitions'').

\textbf{Question.}  Using persistent homology (or another topological data
analysis method), characterise the Betti numbers $\beta_k$ of the point
cloud $\{\Sigma_t\}$ as a function of market state.  Does $\beta_1 > 0$
(the presence of loops in the covariance trajectory) predict volatility
regime changes?  Is there a Morse-theoretic interpretation in terms of
critical points of the geodesic distance function on $\mathcal{P}_n$?
\end{openquestion}

