%% 02_introduction.tex — Introduction
\section{Introduction}
\label{sec:introduction}

\subsection{The Flat-Time Assumption in Classical Finance}

Standard financial mathematics inherits, largely without scrutiny, a Galilean
spacetime structure.  The Black-Scholes framework~\cite{Black1973} posits a
universal, shared clock: all assets evolve against the same parameter $t$,
price paths are Markovian with respect to that clock, and the volatility
$\sigma$ is a scalar field on $\mathbb{R}_+ \times \mathbb{R}$ independent of
observer velocity.  This is the finance equivalent of Newtonian absolute time.

The assumption is load-bearing.  It underlies the no-arbitrage condition,
risk-neutral pricing, and the Feynman-Ka\v{c} representation of derivative
prices.  Yet it fails in precisely the regimes that matter most: high-frequency
microstructure, where latency asymmetries create effective time dilation
between participants~\cite{WissnerGross2010}; cross-asset dynamics, where
information propagates at finite speed through correlation networks; and
volatility clustering~\cite{Engle1982}, where the local ``pace'' of price
evolution differs systematically from the clock on the wall.

A market participant whose quote-update latency is $\Delta t_1$ and whose
counterparty latency is $\Delta t_2 \neq \Delta t_1$ cannot, in general, agree
on whether two events are simultaneous.  This is not a failure of technology;
it is a structural feature of any system in which signal propagation is
bounded.

\subsection{Prior Work and Its Limits}

The econophysics literature has taken four significant steps toward
formalizing this intuition, which we review in detail in
Section~\ref{sec:related}.  In brief:

\begin{enumerate}
  \item \textbf{Wissner-Gross \& Freer} (2010) showed that optimal
        arbitrage nodes between exchange pairs lie on the light-cone of a
        relativistic spacetime defined by signal propagation speed.  Their
        result is geometric and location-theoretic; it does not operate on
        price time series.

  \item \textbf{Kakushadze} (2017) applied Lorentz boosts to the
        volatility smile, deriving a covariant Black-Scholes equation.  The
        framework is elegant but requires $\beta$ as an exogenous parameter;
        no method is given for computing it from \ohlcv{} data.

  \item \textbf{Romero \& Zubieta-Martinez} (2016) replaced the
        Schrödinger equation underlying quantum finance with the Dirac
        equation, yielding a relativistic option pricing formula.  The work
        is purely theoretical with no empirical component.

  \item \textbf{Carvalho \& Gaspar} (2021) defined a Riemannian metric on
        price space and derived covariant stochastic differential equations.
        They identify the metric tensor and its role in geodesic pricing, but
        do not compute Christoffel symbols or geodesic solutions from data.
\end{enumerate}

\noindent
The gap is uniform: prior work establishes that relativistic geometry
\emph{applies} to financial markets but delivers no operational system for
computing relativistic quantities from the \ohlcv{} bars that practitioners
actually observe.  We fill that gap.

\subsection{Contributions}

This paper makes five distinct contributions:

\paragraph{C1 — Operational $\pbeta/\pgamma$ framework.}
We define price velocity $\pbeta$ as a dimensionless ratio of bar-to-bar
price displacement to a calibrated market speed limit $\cmarket$, derive the
Lorentz factor $\pgamma$ from it, and implement both as online (streaming)
estimators over rolling \ohlcv{} windows.  The estimators handle the boundary
case $|\pbeta| \to 1$ via L'Hôpital regularisation without requiring
exception handling or IEEE754 infinity propagation.

\paragraph{C2 — Spacetime interval regime classifier.}
We compute the Minkowski spacetime interval
$\sinterval^2 = c^2 \Delta t^2 - \Delta x^2$ on a discretised
price-time grid, partitioning bars into \textsc{timelike}
($\sinterval^2 > 0$), \textsc{lightlike} ($\sinterval^2 = 0$), and
\textsc{spacelike} ($\sinterval^2 < 0$) regimes.  We show empirically
that the conditional return distribution differs significantly across
regimes~\cite{AGT07results}.

\paragraph{C3 — Christoffel symbols on the covariance manifold.}
We treat the rolling empirical covariance matrix of price features as a point
on the symmetric positive-definite (SPD) manifold $\mathcal{P}_n$, equipped
with the Fisher-Rao metric.  We derive and implement the Christoffel symbols
$\chris{\lambda}{\mu}{\nu}$ analytically from the metric and its first
derivatives, computed numerically via finite differences on the SPD manifold.

\paragraph{C4 — Geodesic deviation signal.}
We solve the geodesic deviation equation
$\frac{D^2 \geovec}{d\tproper^2} + R^{\mu}{}_{\nu\rho\sigma} U^{\nu} \dot{U}^{\rho} U^{\sigma} = 0$
for nearby covariance trajectories, and use the norm of the deviation vector
$\|\geovec\|$ as a regime-change signal.  Large geodesic deviation indicates
that the market's covariance structure is diverging from its geodesic path---a
precursor to volatility regime shifts.

\paragraph{C5 — Empirical Q1 validation.}
We apply the \textsc{srfm} framework to a universe of US equity \ohlcv{} bars
spanning Q1 2025 and report: (i)~the variance ratio between \textsc{timelike}
and \textsc{spacelike} return distributions; (ii)~backtested Sharpe ratio
improvement from conditioning on geodesic deviation percentile; (iii)~Kolmogorov-Smirnov
statistics for regime separation~\cite{AGT07results}.

\subsection{Paper Organisation}

Section~\ref{sec:related} surveys prior work and presents a comparison matrix.
Section~\ref{sec:framework} develops the full mathematical framework with
derivations.  Section~\ref{sec:implementation} describes the C++20 architecture.
Section~\ref{sec:empirical} presents empirical results.
Section~\ref{sec:openquestions} formalises six open problems.
Section~\ref{sec:conclusion} concludes.

All source code, test suites, and figure-generation scripts are available in
the accompanying repository.  The implementation enforces a $\geq 1.5:1$
test-to-production line ratio and zero-panic guarantees on all production code
paths.

