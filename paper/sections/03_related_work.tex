%% 03_related_work.tex — Related Work
\section{Related Work}
\label{sec:related}

\subsection{Wissner-Gross \& Freer (2010): Relativistic Statistical Arbitrage}

Wissner-Gross and Freer~\cite{WissnerGross2010} introduced the concept of
\emph{relativistic statistical arbitrage} by observing that the light-cone
structure of special relativity constrains which exchange pairs can engage in
latency-sensitive arbitrage.  Let two exchanges be located at spacetime
coordinates $(t_1, \mathbf{x}_1)$ and $(t_2, \mathbf{x}_2)$.  An arbitrage
signal can propagate between them only if the spacetime interval satisfies
\begin{equation}
  \Delta s^2 = c^2 (t_2 - t_1)^2 - |\mathbf{x}_2 - \mathbf{x}_1|^2 > 0,
\end{equation}
where $c$ is the speed of light (or electromagnetic signal propagation speed
through fibre).  Their key result is that optimal arbitrage relay nodes lie on
the hyperboloid of constant proper time between the two exchanges.

\textbf{What they did not do.}  Wissner-Gross and Freer operate entirely in
physical spacetime; their ``price'' is the trigger for a binary arbitrage
decision, not a quantity with its own relativistic kinematics.  They do not
define price velocity, Lorentz factor, or any relativistic transformation of
the price time series itself.  Their framework cannot be applied to a single
\ohlcv{} bar.

\subsection{Kakushadze (2017): Volatility Smile as Relativistic Effect}

Kakushadze~\cite{Kakushadze2017} observed that the implied volatility smile
admits a natural interpretation as a Lorentz contraction effect.  Starting
from the covariant generalisation of the Black-Scholes PDE, he derives a
boost-covariant option pricing formula in which the volatility parameter
transforms as
\begin{equation}
  \sigma' = \sigma \sqrt{1 - \beta^2} = \frac{\sigma}{\pgamma},
  \label{eq:kakushadze_vol}
\end{equation}
where $\beta$ is the ``price velocity'' of the underlying in some reference
frame.  The smile arises because different strikes correspond to different
effective velocities relative to the at-the-money frame.

\textbf{What he did not do.}  Kakushadze treats $\beta$ as a free parameter to
be inferred from option market data (the smile itself), not as a quantity
computable from the underlying's \ohlcv{} time series.  The Lorentz factor
$\pgamma$ is thus a fit parameter, not a measurement.  No Christoffel symbols,
geodesics, or spacetime intervals are computed.  The framework is self-contained
within the options domain and requires option market access that is unavailable
for most assets.

\subsection{Romero \& Zubieta-Martinez (2016): Relativistic Quantum Finance}

Romero and Zubieta-Martinez~\cite{Romero2016} extended the quantum finance
programme of Baaquie by replacing the non-relativistic Schrödinger equation
with the relativistic Dirac equation.  The Dirac Hamiltonian for a ``price
spinor'' $\Psi$ is
\begin{equation}
  i\hbar \frac{\partial \Psi}{\partial t} = \left(
    -ic\hbar \boldsymbol{\alpha} \cdot \nabla_x + mc^2 \beta_D
  \right) \Psi,
\end{equation}
where $\boldsymbol{\alpha}$ and $\beta_D$ are Dirac matrices, and the mass $m$
encodes the mean-reverting strength of the asset.  The resulting option pricing
formula contains additional spinor correction terms relative to
Black-Scholes.

\textbf{What they did not do.}  This work is entirely theoretical; no
empirical test is performed.  The connection between the ``price spinor'' and
observable \ohlcv{} quantities is not established.  The framework requires
quantum field-theoretic machinery that has no clear operational analogue in a
trading system.  No Lorentz factors, spacetime intervals, or geodesics appear.

\subsection{Carvalho \& Gaspar (2021): Relativistic Option Pricing}

Carvalho and Gaspar~\cite{Carvalho2021} take the most geometrically
sophisticated approach among the four: they equip the price manifold $\mathcal{M}$
with a Riemannian metric $g_{\mu\nu}$ and derive the covariant SDE for price
dynamics,
\begin{equation}
  dS^\mu + \chris{\mu}{\nu}{\rho} S^\nu \, dS^\rho
  = \mu^\mu \, dt + \sigma^\mu{}_\nu \, dW^\nu,
  \label{eq:carvalho_sde}
\end{equation}
where the Christoffel symbols account for curvature in the price manifold.
They show that geodesic pricing corresponds to the minimum-entropy risk-neutral
measure, and derive a covariant Black-Scholes formula.

\textbf{What they did not do.}  Carvalho and Gaspar define the metric $g_{\mu\nu}$
as a theoretical object associated with the chosen parameterisation of price
space; they do not specify how to \emph{compute} $g_{\mu\nu}$ or the Christoffel
symbols from observed \ohlcv{} data.  The geodesic equation
\eqref{eq:carvalho_sde} is written but not solved numerically.  No empirical
validation is provided.  In particular, the Christoffel symbols $\chris{\mu}{\nu}{\rho}$
in Eq.~\eqref{eq:carvalho_sde} are treated as formal objects, not computable
quantities.

\subsection{Comparison Matrix}

Table~\ref{tab:prior_work} summarises the capabilities of each prior work
relative to the present contribution.

\begin{table}[ht]
\centering
\caption{%
  Prior work comparison. \checkmark~= present and operational;
  $\circ$~= present but as a free parameter or theoretical object;
  $\times$~= absent.
}
\label{tab:prior_work}
\setlength{\tabcolsep}{6pt}
\begin{tabular}{lccccc}
  \toprule
  \textbf{Capability} &
  \rotatebox{70}{\parbox{2.5cm}{\centering WG\&F\\(2010)}} &
  \rotatebox{70}{\parbox{2.5cm}{\centering Kakushadze\\(2017)}} &
  \rotatebox{70}{\parbox{2.5cm}{\centering R\&ZM\\(2016)}} &
  \rotatebox{70}{\parbox{2.5cm}{\centering C\&G\\(2021)}} &
  \rotatebox{70}{\parbox{2.5cm}{\centering \textbf{SRFM}\\(2026)}} \\
  \midrule
  Price velocity $\pbeta$ from \ohlcv{}       & $\times$ & $\circ$ & $\times$ & $\times$ & \checkmark \\
  Lorentz factor $\pgamma$ computed            & $\times$ & $\circ$ & $\times$ & $\times$ & \checkmark \\
  Spacetime interval classifier                & $\times$ & $\times$ & $\times$ & $\times$ & \checkmark \\
  Metric tensor $g_{\mu\nu}$ from data         & $\times$ & $\times$ & $\times$ & $\circ$  & \checkmark \\
  Christoffel symbols $\Gamma$ computed        & $\times$ & $\times$ & $\times$ & $\circ$  & \checkmark \\
  Geodesic equation solved                     & $\times$ & $\times$ & $\times$ & $\circ$  & \checkmark \\
  Geodesic deviation signal                    & $\times$ & $\times$ & $\times$ & $\times$ & \checkmark \\
  Rapidity / velocity composition              & $\times$ & $\times$ & $\times$ & $\times$ & \checkmark \\
  Empirical validation on \ohlcv{} data        & $\times$ & $\times$ & $\times$ & $\times$ & \checkmark \\
  C++ / production implementation              & $\times$ & $\times$ & $\times$ & $\times$ & \checkmark \\
  \bottomrule
\end{tabular}
\end{table}

\subsection{Information Geometry and SPD Manifolds}

Our treatment of the covariance manifold (Section~\ref{sec:framework},
§\ref{subsec:manifold}) draws on the information geometry
literature~\cite{Amari1998, Pennec2006, Nielsen2020}.  The symmetric
positive-definite (SPD) manifold $\mathcal{P}_n$ with the
affine-invariant (Fisher-Rao) metric is well-studied in the context of
diffusion tensor imaging and radar covariance estimation; we adapt these
tools to the financial covariance matrix of rolling price features.

\subsection{Positioning Relative to the Broader Econophysics Programme}

Our work sits at the intersection of three active research programmes:
(i)~econophysics, which applies statistical mechanics and field theory to
financial markets~\cite{Mantegna1999, Bouchaud2003};
(ii)~information geometry applied to finance~\cite{Amari1998}; and
(iii)~the emerging relativistic finance literature surveyed above.  We are
the first to bring all three together in an operational system.

