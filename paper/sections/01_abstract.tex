%% 01_abstract.tex — Abstract
\begin{abstract}

Classical finance theory assumes a flat, Galilean time structure in which all
market participants share a universal clock and price velocity composes
linearly.  Four prior works---Wissner-Gross \& Freer (2010), Kakushadze
(2017), Romero \& Zubieta-Martinez (2016), and Carvalho \& Gaspar
(2021)---demonstrated that special-relativistic geometry provides a
mathematically consistent and empirically motivated alternative.  Each of
these contributions, however, remained at the level of theoretical framework:
no prior work delivered an operational system that computes Lorentz factors,
spacetime intervals, Christoffel symbols, or geodesic deviations from live
\ohlcv{} bar data.

We close this gap.  We present the Special-Relativistic Finance Manifold
(\textsc{srfm}), the first end-to-end C++20 implementation of a
special-relativistic price dynamics framework.  Our contributions are
fivefold: (i)~an operational price-velocity $\pbeta$ and Lorentz-factor
$\pgamma$ calculator derived from open--high--low--close--volume bars; (ii)~a
spacetime interval classifier that partitions market regimes into
\textsc{timelike} and \textsc{spacelike} sectors analogous to causal cones in
special relativity; (iii)~a Christoffel symbol computation on the empirical
covariance manifold spanned by rolling price features; (iv)~a geodesic
deviation solver whose output functions as a regime-change signal; and (v)~an
empirical validation on Q1 2025 equity data demonstrating that
\textsc{timelike} bars exhibit statistically distinct return variance from
\textsc{spacelike} bars~\cite{AGT07results}.  All five components are
implemented with zero-panic guarantees, strong-typed interfaces, and a
$1.5:1$ test-to-production line ratio enforced by continuous integration.

\end{abstract}

